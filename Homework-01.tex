% Options for packages loaded elsewhere
\PassOptionsToPackage{unicode}{hyperref}
\PassOptionsToPackage{hyphens}{url}
%
\documentclass[
]{article}
\usepackage{lmodern}
\usepackage{amssymb,amsmath}
\usepackage{ifxetex,ifluatex}
\ifnum 0\ifxetex 1\fi\ifluatex 1\fi=0 % if pdftex
  \usepackage[T1]{fontenc}
  \usepackage[utf8]{inputenc}
  \usepackage{textcomp} % provide euro and other symbols
\else % if luatex or xetex
  \usepackage{unicode-math}
  \defaultfontfeatures{Scale=MatchLowercase}
  \defaultfontfeatures[\rmfamily]{Ligatures=TeX,Scale=1}
\fi
% Use upquote if available, for straight quotes in verbatim environments
\IfFileExists{upquote.sty}{\usepackage{upquote}}{}
\IfFileExists{microtype.sty}{% use microtype if available
  \usepackage[]{microtype}
  \UseMicrotypeSet[protrusion]{basicmath} % disable protrusion for tt fonts
}{}
\makeatletter
\@ifundefined{KOMAClassName}{% if non-KOMA class
  \IfFileExists{parskip.sty}{%
    \usepackage{parskip}
  }{% else
    \setlength{\parindent}{0pt}
    \setlength{\parskip}{6pt plus 2pt minus 1pt}}
}{% if KOMA class
  \KOMAoptions{parskip=half}}
\makeatother
\usepackage{xcolor}
\IfFileExists{xurl.sty}{\usepackage{xurl}}{} % add URL line breaks if available
\IfFileExists{bookmark.sty}{\usepackage{bookmark}}{\usepackage{hyperref}}
\hypersetup{
  pdftitle={Homework 1},
  hidelinks,
  pdfcreator={LaTeX via pandoc}}
\urlstyle{same} % disable monospaced font for URLs
\usepackage[margin=1in]{geometry}
\usepackage{color}
\usepackage{fancyvrb}
\newcommand{\VerbBar}{|}
\newcommand{\VERB}{\Verb[commandchars=\\\{\}]}
\DefineVerbatimEnvironment{Highlighting}{Verbatim}{commandchars=\\\{\}}
% Add ',fontsize=\small' for more characters per line
\usepackage{framed}
\definecolor{shadecolor}{RGB}{248,248,248}
\newenvironment{Shaded}{\begin{snugshade}}{\end{snugshade}}
\newcommand{\AlertTok}[1]{\textcolor[rgb]{0.94,0.16,0.16}{#1}}
\newcommand{\AnnotationTok}[1]{\textcolor[rgb]{0.56,0.35,0.01}{\textbf{\textit{#1}}}}
\newcommand{\AttributeTok}[1]{\textcolor[rgb]{0.77,0.63,0.00}{#1}}
\newcommand{\BaseNTok}[1]{\textcolor[rgb]{0.00,0.00,0.81}{#1}}
\newcommand{\BuiltInTok}[1]{#1}
\newcommand{\CharTok}[1]{\textcolor[rgb]{0.31,0.60,0.02}{#1}}
\newcommand{\CommentTok}[1]{\textcolor[rgb]{0.56,0.35,0.01}{\textit{#1}}}
\newcommand{\CommentVarTok}[1]{\textcolor[rgb]{0.56,0.35,0.01}{\textbf{\textit{#1}}}}
\newcommand{\ConstantTok}[1]{\textcolor[rgb]{0.00,0.00,0.00}{#1}}
\newcommand{\ControlFlowTok}[1]{\textcolor[rgb]{0.13,0.29,0.53}{\textbf{#1}}}
\newcommand{\DataTypeTok}[1]{\textcolor[rgb]{0.13,0.29,0.53}{#1}}
\newcommand{\DecValTok}[1]{\textcolor[rgb]{0.00,0.00,0.81}{#1}}
\newcommand{\DocumentationTok}[1]{\textcolor[rgb]{0.56,0.35,0.01}{\textbf{\textit{#1}}}}
\newcommand{\ErrorTok}[1]{\textcolor[rgb]{0.64,0.00,0.00}{\textbf{#1}}}
\newcommand{\ExtensionTok}[1]{#1}
\newcommand{\FloatTok}[1]{\textcolor[rgb]{0.00,0.00,0.81}{#1}}
\newcommand{\FunctionTok}[1]{\textcolor[rgb]{0.00,0.00,0.00}{#1}}
\newcommand{\ImportTok}[1]{#1}
\newcommand{\InformationTok}[1]{\textcolor[rgb]{0.56,0.35,0.01}{\textbf{\textit{#1}}}}
\newcommand{\KeywordTok}[1]{\textcolor[rgb]{0.13,0.29,0.53}{\textbf{#1}}}
\newcommand{\NormalTok}[1]{#1}
\newcommand{\OperatorTok}[1]{\textcolor[rgb]{0.81,0.36,0.00}{\textbf{#1}}}
\newcommand{\OtherTok}[1]{\textcolor[rgb]{0.56,0.35,0.01}{#1}}
\newcommand{\PreprocessorTok}[1]{\textcolor[rgb]{0.56,0.35,0.01}{\textit{#1}}}
\newcommand{\RegionMarkerTok}[1]{#1}
\newcommand{\SpecialCharTok}[1]{\textcolor[rgb]{0.00,0.00,0.00}{#1}}
\newcommand{\SpecialStringTok}[1]{\textcolor[rgb]{0.31,0.60,0.02}{#1}}
\newcommand{\StringTok}[1]{\textcolor[rgb]{0.31,0.60,0.02}{#1}}
\newcommand{\VariableTok}[1]{\textcolor[rgb]{0.00,0.00,0.00}{#1}}
\newcommand{\VerbatimStringTok}[1]{\textcolor[rgb]{0.31,0.60,0.02}{#1}}
\newcommand{\WarningTok}[1]{\textcolor[rgb]{0.56,0.35,0.01}{\textbf{\textit{#1}}}}
\usepackage{graphicx,grffile}
\makeatletter
\def\maxwidth{\ifdim\Gin@nat@width>\linewidth\linewidth\else\Gin@nat@width\fi}
\def\maxheight{\ifdim\Gin@nat@height>\textheight\textheight\else\Gin@nat@height\fi}
\makeatother
% Scale images if necessary, so that they will not overflow the page
% margins by default, and it is still possible to overwrite the defaults
% using explicit options in \includegraphics[width, height, ...]{}
\setkeys{Gin}{width=\maxwidth,height=\maxheight,keepaspectratio}
% Set default figure placement to htbp
\makeatletter
\def\fps@figure{htbp}
\makeatother
\setlength{\emergencystretch}{3em} % prevent overfull lines
\providecommand{\tightlist}{%
  \setlength{\itemsep}{0pt}\setlength{\parskip}{0pt}}
\setcounter{secnumdepth}{-\maxdimen} % remove section numbering

\title{Homework 1}
\author{}
\date{\vspace{-2.5em}}

\begin{document}
\maketitle

\begin{enumerate}
\def\labelenumi{\arabic{enumi}.}
\tightlist
\item
  The Iowa data set iowa.csv is a toy example that summarises the yield
  of wheat (bushels per acre) for the state of Iowa between 1930-1962.
  In addition to yield, year, rainfall and temperature were recorded as
  the main predictors of yield.

  \begin{enumerate}
  \def\labelenumii{\alph{enumii}.}
  \tightlist
  \item
    First, we need to load the data set into R using the command
    \texttt{read.csv()}. Use the help function to learn what arguments
    this function takes. Once you have the necessary input, load the
    data set into R and make it a data frame called \texttt{iowa.df}.
  \end{enumerate}
\end{enumerate}

\begin{Shaded}
\begin{Highlighting}[]
\NormalTok{iowa.df<-}\KeywordTok{read.csv}\NormalTok{(}\StringTok{"data/iowa.csv"}\NormalTok{,}\DataTypeTok{header =} \OtherTok{TRUE}\NormalTok{,}\DataTypeTok{sep =} \StringTok{';'}\NormalTok{)}
\end{Highlighting}
\end{Shaded}

\begin{verbatim}
b. How many rows and columns does `iowa.df` have? 
   answer:33 rows and 10 columns
\end{verbatim}

\begin{Shaded}
\begin{Highlighting}[]
\KeywordTok{dim}\NormalTok{(iowa.df)}
\end{Highlighting}
\end{Shaded}

\begin{verbatim}
## [1] 33 10
\end{verbatim}

\begin{verbatim}
c. What are the names of the columns of `iowa.df`? 
   answer: Year Rain0 Temp1 Rain1 Temp2 Rain2 Temp3 Rain3 Temp4 Yield
\end{verbatim}

\begin{Shaded}
\begin{Highlighting}[]
\KeywordTok{str}\NormalTok{(iowa.df)}
\end{Highlighting}
\end{Shaded}

\begin{verbatim}
## 'data.frame':    33 obs. of  10 variables:
##  $ Year : int  1930 1931 1932 1933 1934 1935 1936 1937 1938 1939 ...
##  $ Rain0: num  17.8 14.8 28 16.8 11.4 ...
##  $ Temp1: num  60.2 57.5 62.3 60.5 69.5 55 66.2 61.8 59.5 66.4 ...
##  $ Rain1: num  5.83 3.83 5.17 1.64 3.49 7 2.85 3.8 4.67 5.32 ...
##  $ Temp2: num  69 75 72 77.8 77.2 65.9 70.1 69 69.2 71.4 ...
##  $ Rain2: num  1.49 2.72 3.12 3.45 3.85 3.35 0.51 2.63 4.24 3.15 ...
##  $ Temp3: num  77.9 77.2 75.8 76.4 79.7 79.4 83.4 75.9 76.5 76.2 ...
##  $ Rain3: num  2.42 3.3 7.1 3.01 2.84 2.42 3.48 3.99 3.82 4.72 ...
##  $ Temp4: num  74.4 72.6 72.2 70.5 73.4 73.6 79.2 77.8 75.7 70.7 ...
##  $ Yield: num  34 32.9 43 40 23 38.4 20 44.6 46.3 52.2 ...
\end{verbatim}

\begin{verbatim}
d. What is the value of row 5, column 7 of `iowa.df`?
   answer: 79.7
\end{verbatim}

\begin{Shaded}
\begin{Highlighting}[]
\NormalTok{iowa.df[}\DecValTok{5}\NormalTok{,}\DecValTok{7}\NormalTok{]}
\end{Highlighting}
\end{Shaded}

\begin{verbatim}
## [1] 79.7
\end{verbatim}

\begin{verbatim}
e. Display the second row of `iowa.df` in its entirety.
\end{verbatim}

\begin{Shaded}
\begin{Highlighting}[]
\NormalTok{iowa.df[}\DecValTok{2}\NormalTok{,]}
\end{Highlighting}
\end{Shaded}

\begin{verbatim}
##   Year Rain0 Temp1 Rain1 Temp2 Rain2 Temp3 Rain3 Temp4 Yield
## 2 1931 14.76  57.5  3.83    75  2.72  77.2   3.3  72.6  32.9
\end{verbatim}

\begin{enumerate}
\def\labelenumi{\arabic{enumi}.}
\setcounter{enumi}{1}
\item
  Syntax and class-typing.

  \begin{enumerate}
  \def\labelenumii{\alph{enumii}.}
  \tightlist
  \item
    For each of the following commands, either explain why they should
    be errors, or explain the non-erroneous result
  \end{enumerate}

\begin{verbatim}
vector1 <- c("5", "12", "7", "32") 
answer:"5"  "12" "7"  "32"
max(vector1)  
answer:"7"
sort(vector1) 
answer:"12" "32" "5"  "7" 
sum(vector1) 
answer:the data structure is chr
\end{verbatim}

  \begin{enumerate}
  \def\labelenumii{\alph{enumii}.}
  \setcounter{enumii}{1}
  \tightlist
  \item
    For the next series of commands, either explain their results, or
    why they should produce errors.
  \end{enumerate}
\end{enumerate}

\begin{verbatim}
vector2 <- c("5",7,12) 
answer:"5"  "7"  "12"
vector2[2] + vector2[3] 
answer:Error because the default is to character type
dataframe3 <- data.frame(z1="5",z2=7,z3=12) 
answer:  z1 z2 z3
         5  7 12
dataframe3[1,2] + dataframe3[1,3]
answer: 19
list4 <- list(z1="6", z2=42, z3="49", z4=126)
answer:
$z1
[1] "6"

$z2
[1] 42

$z3
[1] "49"

$z4
[1] 126
list4[[2]]+list4[[4]]
answer:168
list4[2]+list4[4]
answer:list4[2]=z2,list4[4]=z4,cannot calculate
\end{verbatim}

\begin{enumerate}
\def\labelenumi{\arabic{enumi}.}
\setcounter{enumi}{2}
\tightlist
\item
  Working with functions and operators.

  \begin{enumerate}
  \def\labelenumii{\alph{enumii}.}
  \tightlist
  \item
    The colon operator will create a sequence of integers in order. It
    is a special case of the function \texttt{seq()} which you saw
    earlier in this assignment. Using the help command \texttt{?seq} to
    learn about the function, design an expression that will give you
    the sequence of numbers from 1 to 10000 in increments of 372. Design
    another that will give you a sequence between 1 and 10000 that is
    exactly 50 numbers in length.
  \end{enumerate}
\end{enumerate}

\begin{Shaded}
\begin{Highlighting}[]
\KeywordTok{seq}\NormalTok{(}\DecValTok{1}\NormalTok{,}\DecValTok{1000}\NormalTok{,}\DecValTok{372}\NormalTok{)}
\end{Highlighting}
\end{Shaded}

\begin{verbatim}
## [1]   1 373 745
\end{verbatim}

\begin{Shaded}
\begin{Highlighting}[]
\KeywordTok{seq}\NormalTok{(}\DecValTok{1}\NormalTok{,}\DecValTok{1000}\NormalTok{,}\DataTypeTok{length.out =} \DecValTok{50}\NormalTok{)}
\end{Highlighting}
\end{Shaded}

\begin{verbatim}
##  [1]    1.00000   21.38776   41.77551   62.16327   82.55102  102.93878
##  [7]  123.32653  143.71429  164.10204  184.48980  204.87755  225.26531
## [13]  245.65306  266.04082  286.42857  306.81633  327.20408  347.59184
## [19]  367.97959  388.36735  408.75510  429.14286  449.53061  469.91837
## [25]  490.30612  510.69388  531.08163  551.46939  571.85714  592.24490
## [31]  612.63265  633.02041  653.40816  673.79592  694.18367  714.57143
## [37]  734.95918  755.34694  775.73469  796.12245  816.51020  836.89796
## [43]  857.28571  877.67347  898.06122  918.44898  938.83673  959.22449
## [49]  979.61224 1000.00000
\end{verbatim}

\begin{verbatim}
b. The function `rep()` repeats a vector some number of times. Explain the difference between `rep(1:3, times=3) and rep(1:3, each=3).
answer:One is to circulate 3 times according to 1.2.3; The other is to circulate for 3 times according to 1.2.3 Respectively
\end{verbatim}

\begin{Shaded}
\begin{Highlighting}[]
\KeywordTok{rep}\NormalTok{(}\DecValTok{1}\OperatorTok{:}\DecValTok{3}\NormalTok{, }\DataTypeTok{times=}\DecValTok{3}\NormalTok{)}
\end{Highlighting}
\end{Shaded}

\begin{verbatim}
## [1] 1 2 3 1 2 3 1 2 3
\end{verbatim}

\begin{Shaded}
\begin{Highlighting}[]
\KeywordTok{rep}\NormalTok{(}\DecValTok{1}\OperatorTok{:}\DecValTok{3}\NormalTok{, }\DataTypeTok{each=}\DecValTok{3}\NormalTok{)}
\end{Highlighting}
\end{Shaded}

\begin{verbatim}
## [1] 1 1 1 2 2 2 3 3 3
\end{verbatim}

MB.Ch1.2. The orings data frame gives data on the damage that had
occurred in US space shuttle launches prior to the disastrous Challenger
launch of 28 January 1986. The observations in rows 1, 2, 4, 11, 13,and
18 were included in the pre-launch charts used in deciding whether to
proceed with the launch, while remaining rows were omitted.

Create a new data frame by extracting these rows from orings, and plot
total incidents against temperature for this new data frame. Obtain a
similar plot for the full data set.

\begin{Shaded}
\begin{Highlighting}[]
\NormalTok{orings_new=orings[}\KeywordTok{c}\NormalTok{(}\DecValTok{1}\NormalTok{,}\DecValTok{2}\NormalTok{,}\DecValTok{4}\NormalTok{,}\DecValTok{11}\NormalTok{,}\DecValTok{13}\NormalTok{,}\DecValTok{18}\NormalTok{),]}
\KeywordTok{plot}\NormalTok{(orings_new}\OperatorTok{$}\NormalTok{Temperature,orings_new}\OperatorTok{$}\NormalTok{Total)}
\end{Highlighting}
\end{Shaded}

\includegraphics{Homework-01_files/figure-latex/unnamed-chunk-8-1.pdf}

\begin{Shaded}
\begin{Highlighting}[]
\KeywordTok{plot}\NormalTok{(orings}\OperatorTok{$}\NormalTok{Temperature,orings}\OperatorTok{$}\NormalTok{Total)}
\end{Highlighting}
\end{Shaded}

\includegraphics{Homework-01_files/figure-latex/unnamed-chunk-8-2.pdf}
MB.Ch1.4. For the data frame ais (DAAG package)

\begin{enumerate}
\def\labelenumi{(\alph{enumi})}
\tightlist
\item
  Use the function str() to get information on each of the columns.
  Determine whether any of the columns hold missing values.
\end{enumerate}

\begin{Shaded}
\begin{Highlighting}[]
\KeywordTok{str}\NormalTok{(ais)}
\end{Highlighting}
\end{Shaded}

\begin{verbatim}
## 'data.frame':    202 obs. of  13 variables:
##  $ rcc   : num  3.96 4.41 4.14 4.11 4.45 4.1 4.31 4.42 4.3 4.51 ...
##  $ wcc   : num  7.5 8.3 5 5.3 6.8 4.4 5.3 5.7 8.9 4.4 ...
##  $ hc    : num  37.5 38.2 36.4 37.3 41.5 37.4 39.6 39.9 41.1 41.6 ...
##  $ hg    : num  12.3 12.7 11.6 12.6 14 12.5 12.8 13.2 13.5 12.7 ...
##  $ ferr  : num  60 68 21 69 29 42 73 44 41 44 ...
##  $ bmi   : num  20.6 20.7 21.9 21.9 19 ...
##  $ ssf   : num  109.1 102.8 104.6 126.4 80.3 ...
##  $ pcBfat: num  19.8 21.3 19.9 23.7 17.6 ...
##  $ lbm   : num  63.3 58.5 55.4 57.2 53.2 ...
##  $ ht    : num  196 190 178 185 185 ...
##  $ wt    : num  78.9 74.4 69.1 74.9 64.6 63.7 75.2 62.3 66.5 62.9 ...
##  $ sex   : Factor w/ 2 levels "f","m": 1 1 1 1 1 1 1 1 1 1 ...
##  $ sport : Factor w/ 10 levels "B_Ball","Field",..: 1 1 1 1 1 1 1 1 1 1 ...
\end{verbatim}

\begin{enumerate}
\def\labelenumi{(\alph{enumi})}
\setcounter{enumi}{1}
\tightlist
\item
  Make a table that shows the numbers of males and females for each
  different sport. In which sports is there a large imbalance (e.g., by
  a factor of more than 2:1) in the numbers of the two sexes?
  answer:Gym,Netball,T\_sprnt,W\_polo have a large imbalance
\end{enumerate}

\begin{Shaded}
\begin{Highlighting}[]
\NormalTok{ais }\OperatorTok\StringTok{ }\KeywordTok{group_by}\NormalTok{(sport) }\OperatorTok\StringTok{ }\KeywordTok{summarise}\NormalTok{(}\DataTypeTok{male=}\KeywordTok{sum}\NormalTok{(sex}\OperatorTok{==}\StringTok{'m'}\NormalTok{),}\DataTypeTok{female=}\KeywordTok{sum}\NormalTok{(sex}\OperatorTok{==}\StringTok{'f'}\NormalTok{)) }\OperatorTok\StringTok{ }\KeywordTok{mutate}\NormalTok{(}\DataTypeTok{proportion=}\NormalTok{male}\OperatorTok{/}\NormalTok{female)}
\end{Highlighting}
\end{Shaded}

\begin{verbatim}
## `summarise()` ungrouping output (override with `.groups` argument)
\end{verbatim}

\begin{verbatim}
## # A tibble: 10 x 4
##    sport    male female proportion
##    <fct>   <int>  <int>      <dbl>
##  1 B_Ball     12     13      0.923
##  2 Field      12      7      1.71 
##  3 Gym         0      4      0    
##  4 Netball     0     23      0    
##  5 Row        15     22      0.682
##  6 Swim       13      9      1.44 
##  7 T_400m     18     11      1.64 
##  8 T_Sprnt    11      4      2.75 
##  9 Tennis      4      7      0.571
## 10 W_Polo     17      0    Inf
\end{verbatim}

MB.Ch1.6.Create a data frame called Manitoba.lakes that contains the
lake's elevation (in meters above sea level) and area (in square
kilometers) as listed below. Assign the names of the lakes using the
row.names() function. elevation area Winnipeg 217 24387 Winnipegosis 254
5374 Manitoba 248 4624 SouthernIndian 254 2247 Cedar 253 1353 Island 227
1223 Gods 178 1151 Cross 207 755 Playgreen 217 657

\begin{Shaded}
\begin{Highlighting}[]
\NormalTok{elevation=}\KeywordTok{c}\NormalTok{(}\DecValTok{217}\NormalTok{,}\DecValTok{254}\NormalTok{,}\DecValTok{248}\NormalTok{,}\DecValTok{254}\NormalTok{,}\DecValTok{253}\NormalTok{,}\DecValTok{227}\NormalTok{,}\DecValTok{178}\NormalTok{,}\DecValTok{207}\NormalTok{,}\DecValTok{217}\NormalTok{)}
\NormalTok{area=}\KeywordTok{c}\NormalTok{(}\DecValTok{24387}\NormalTok{,}\DecValTok{5374}\NormalTok{,}\DecValTok{4624}\NormalTok{,}\DecValTok{2247}\NormalTok{,}\DecValTok{1353}\NormalTok{,}\DecValTok{1223}\NormalTok{,}\DecValTok{1151}\NormalTok{,}\DecValTok{755}\NormalTok{,}\DecValTok{657}\NormalTok{)}
\NormalTok{Manitoba.lakes=}\KeywordTok{data.frame}\NormalTok{(elevation,area)}
\KeywordTok{row.names}\NormalTok{(Manitoba.lakes)=}\KeywordTok{c}\NormalTok{(}\StringTok{"Winnipeg"}\NormalTok{,}\StringTok{"Winnipegosis"}\NormalTok{,}\StringTok{"Manitoba"}\NormalTok{,}\StringTok{"SouthernIndian"}\NormalTok{,}\StringTok{"Cedar"}\NormalTok{,}\StringTok{"Island"}\NormalTok{,}\StringTok{"Gods"}\NormalTok{,}\StringTok{"Cross"}\NormalTok{,}\StringTok{"Playgreen"}\NormalTok{)}
\end{Highlighting}
\end{Shaded}

\begin{enumerate}
\def\labelenumi{(\alph{enumi})}
\tightlist
\item
  Use the following code to plot log2(area) versus elevation, adding
  labeling information (there is an extreme value of area that makes a
  logarithmic scale pretty much essential):
\end{enumerate}

\begin{Shaded}
\begin{Highlighting}[]
\KeywordTok{attach}\NormalTok{(Manitoba.lakes)}
\end{Highlighting}
\end{Shaded}

\begin{verbatim}
## The following objects are masked _by_ .GlobalEnv:
## 
##     area, elevation
\end{verbatim}

\begin{Shaded}
\begin{Highlighting}[]
\KeywordTok{plot}\NormalTok{(}\KeywordTok{log2}\NormalTok{(area) }\OperatorTok{~}\StringTok{ }\NormalTok{elevation, }\DataTypeTok{pch=}\DecValTok{16}\NormalTok{, }\DataTypeTok{xlim=}\KeywordTok{c}\NormalTok{(}\DecValTok{170}\NormalTok{,}\DecValTok{280}\NormalTok{))}
\CommentTok{# NB: Doubling the area increases log2(area) by 1.0}
\KeywordTok{text}\NormalTok{(}\KeywordTok{log2}\NormalTok{(area) }\OperatorTok{~}\StringTok{ }\NormalTok{elevation, }\DataTypeTok{labels=}\KeywordTok{row.names}\NormalTok{(Manitoba.lakes), }\DataTypeTok{pos=}\DecValTok{4}\NormalTok{)}
\KeywordTok{text}\NormalTok{(}\KeywordTok{log2}\NormalTok{(area) }\OperatorTok{~}\StringTok{ }\NormalTok{elevation, }\DataTypeTok{labels=}\NormalTok{area, }\DataTypeTok{pos=}\DecValTok{2}\NormalTok{) }
\KeywordTok{title}\NormalTok{(}\StringTok{"Manitoba’s Largest Lakes"}\NormalTok{)}
\end{Highlighting}
\end{Shaded}

\includegraphics{Homework-01_files/figure-latex/unnamed-chunk-12-1.pdf}
Devise captions that explain the labeling on the points and on the
y-axis. It will be necessary to explain how distances on the scale
relate to changes in area.

\begin{enumerate}
\def\labelenumi{(\alph{enumi})}
\setcounter{enumi}{1}
\tightlist
\item
  Repeat the plot and associated labeling, now plotting area versus
  elevation, but specifying log=``y'' in order to obtain a logarithmic
  y-scale.
\end{enumerate}

\begin{Shaded}
\begin{Highlighting}[]
\KeywordTok{plot}\NormalTok{(area }\OperatorTok{~}\StringTok{ }\NormalTok{elevation, }\DataTypeTok{pch=}\DecValTok{16}\NormalTok{, }\DataTypeTok{xlim=}\KeywordTok{c}\NormalTok{(}\DecValTok{170}\NormalTok{,}\DecValTok{280}\NormalTok{), }\DataTypeTok{ylog=}\NormalTok{T)}
\KeywordTok{text}\NormalTok{(area }\OperatorTok{~}\StringTok{ }\NormalTok{elevation, }\DataTypeTok{labels=}\KeywordTok{row.names}\NormalTok{(Manitoba.lakes), }\DataTypeTok{pos=}\DecValTok{4}\NormalTok{, }\DataTypeTok{ylog=}\NormalTok{T)}
\KeywordTok{text}\NormalTok{(area }\OperatorTok{~}\StringTok{ }\NormalTok{elevation, }\DataTypeTok{labels=}\NormalTok{area, }\DataTypeTok{pos=}\DecValTok{2}\NormalTok{, }\DataTypeTok{ylog=}\NormalTok{T) }
\KeywordTok{title}\NormalTok{(}\StringTok{"Manitoba’s Largest Lakes"}\NormalTok{)}
\end{Highlighting}
\end{Shaded}

\includegraphics{Homework-01_files/figure-latex/unnamed-chunk-13-1.pdf}
MB.Ch1.7. Look up the help page for the R function dotchart(). Use this
function to display the areas of the Manitoba lakes (a) on a linear
scale, and (b) on a logarithmic scale. Add, in each case, suitable
labeling information.

\begin{Shaded}
\begin{Highlighting}[]
\KeywordTok{dotchart}\NormalTok{(}\KeywordTok{log2}\NormalTok{(area),}\DataTypeTok{labels =} \KeywordTok{row.names}\NormalTok{(Manitoba.lakes))}
\end{Highlighting}
\end{Shaded}

\includegraphics{Homework-01_files/figure-latex/unnamed-chunk-14-1.pdf}

\begin{Shaded}
\begin{Highlighting}[]
\KeywordTok{dotchart}\NormalTok{(area,}\DataTypeTok{labels =} \KeywordTok{row.names}\NormalTok{(Manitoba.lakes))}
\end{Highlighting}
\end{Shaded}

\includegraphics{Homework-01_files/figure-latex/unnamed-chunk-14-2.pdf}

MB.Ch1.8. Using the sum() function, obtain a lower bound for the area of
Manitoba covered by water.

\begin{Shaded}
\begin{Highlighting}[]
\KeywordTok{sum}\NormalTok{(Manitoba.lakes}\OperatorTok{$}\NormalTok{area)}
\end{Highlighting}
\end{Shaded}

\begin{verbatim}
## [1] 41771
\end{verbatim}

\end{document}
